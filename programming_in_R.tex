% Options for packages loaded elsewhere
\PassOptionsToPackage{unicode}{hyperref}
\PassOptionsToPackage{hyphens}{url}
\PassOptionsToPackage{dvipsnames,svgnames,x11names}{xcolor}
%
\documentclass[
  letterpaper,
  DIV=11,
  numbers=noendperiod]{scrartcl}

\usepackage{amsmath,amssymb}
\usepackage{iftex}
\ifPDFTeX
  \usepackage[T1]{fontenc}
  \usepackage[utf8]{inputenc}
  \usepackage{textcomp} % provide euro and other symbols
\else % if luatex or xetex
  \usepackage{unicode-math}
  \defaultfontfeatures{Scale=MatchLowercase}
  \defaultfontfeatures[\rmfamily]{Ligatures=TeX,Scale=1}
\fi
\usepackage{lmodern}
\ifPDFTeX\else  
    % xetex/luatex font selection
\fi
% Use upquote if available, for straight quotes in verbatim environments
\IfFileExists{upquote.sty}{\usepackage{upquote}}{}
\IfFileExists{microtype.sty}{% use microtype if available
  \usepackage[]{microtype}
  \UseMicrotypeSet[protrusion]{basicmath} % disable protrusion for tt fonts
}{}
\makeatletter
\@ifundefined{KOMAClassName}{% if non-KOMA class
  \IfFileExists{parskip.sty}{%
    \usepackage{parskip}
  }{% else
    \setlength{\parindent}{0pt}
    \setlength{\parskip}{6pt plus 2pt minus 1pt}}
}{% if KOMA class
  \KOMAoptions{parskip=half}}
\makeatother
\usepackage{xcolor}
\setlength{\emergencystretch}{3em} % prevent overfull lines
\setcounter{secnumdepth}{-\maxdimen} % remove section numbering
% Make \paragraph and \subparagraph free-standing
\makeatletter
\ifx\paragraph\undefined\else
  \let\oldparagraph\paragraph
  \renewcommand{\paragraph}{
    \@ifstar
      \xxxParagraphStar
      \xxxParagraphNoStar
  }
  \newcommand{\xxxParagraphStar}[1]{\oldparagraph*{#1}\mbox{}}
  \newcommand{\xxxParagraphNoStar}[1]{\oldparagraph{#1}\mbox{}}
\fi
\ifx\subparagraph\undefined\else
  \let\oldsubparagraph\subparagraph
  \renewcommand{\subparagraph}{
    \@ifstar
      \xxxSubParagraphStar
      \xxxSubParagraphNoStar
  }
  \newcommand{\xxxSubParagraphStar}[1]{\oldsubparagraph*{#1}\mbox{}}
  \newcommand{\xxxSubParagraphNoStar}[1]{\oldsubparagraph{#1}\mbox{}}
\fi
\makeatother

\usepackage{color}
\usepackage{fancyvrb}
\newcommand{\VerbBar}{|}
\newcommand{\VERB}{\Verb[commandchars=\\\{\}]}
\DefineVerbatimEnvironment{Highlighting}{Verbatim}{commandchars=\\\{\}}
% Add ',fontsize=\small' for more characters per line
\usepackage{framed}
\definecolor{shadecolor}{RGB}{241,243,245}
\newenvironment{Shaded}{\begin{snugshade}}{\end{snugshade}}
\newcommand{\AlertTok}[1]{\textcolor[rgb]{0.68,0.00,0.00}{#1}}
\newcommand{\AnnotationTok}[1]{\textcolor[rgb]{0.37,0.37,0.37}{#1}}
\newcommand{\AttributeTok}[1]{\textcolor[rgb]{0.40,0.45,0.13}{#1}}
\newcommand{\BaseNTok}[1]{\textcolor[rgb]{0.68,0.00,0.00}{#1}}
\newcommand{\BuiltInTok}[1]{\textcolor[rgb]{0.00,0.23,0.31}{#1}}
\newcommand{\CharTok}[1]{\textcolor[rgb]{0.13,0.47,0.30}{#1}}
\newcommand{\CommentTok}[1]{\textcolor[rgb]{0.37,0.37,0.37}{#1}}
\newcommand{\CommentVarTok}[1]{\textcolor[rgb]{0.37,0.37,0.37}{\textit{#1}}}
\newcommand{\ConstantTok}[1]{\textcolor[rgb]{0.56,0.35,0.01}{#1}}
\newcommand{\ControlFlowTok}[1]{\textcolor[rgb]{0.00,0.23,0.31}{\textbf{#1}}}
\newcommand{\DataTypeTok}[1]{\textcolor[rgb]{0.68,0.00,0.00}{#1}}
\newcommand{\DecValTok}[1]{\textcolor[rgb]{0.68,0.00,0.00}{#1}}
\newcommand{\DocumentationTok}[1]{\textcolor[rgb]{0.37,0.37,0.37}{\textit{#1}}}
\newcommand{\ErrorTok}[1]{\textcolor[rgb]{0.68,0.00,0.00}{#1}}
\newcommand{\ExtensionTok}[1]{\textcolor[rgb]{0.00,0.23,0.31}{#1}}
\newcommand{\FloatTok}[1]{\textcolor[rgb]{0.68,0.00,0.00}{#1}}
\newcommand{\FunctionTok}[1]{\textcolor[rgb]{0.28,0.35,0.67}{#1}}
\newcommand{\ImportTok}[1]{\textcolor[rgb]{0.00,0.46,0.62}{#1}}
\newcommand{\InformationTok}[1]{\textcolor[rgb]{0.37,0.37,0.37}{#1}}
\newcommand{\KeywordTok}[1]{\textcolor[rgb]{0.00,0.23,0.31}{\textbf{#1}}}
\newcommand{\NormalTok}[1]{\textcolor[rgb]{0.00,0.23,0.31}{#1}}
\newcommand{\OperatorTok}[1]{\textcolor[rgb]{0.37,0.37,0.37}{#1}}
\newcommand{\OtherTok}[1]{\textcolor[rgb]{0.00,0.23,0.31}{#1}}
\newcommand{\PreprocessorTok}[1]{\textcolor[rgb]{0.68,0.00,0.00}{#1}}
\newcommand{\RegionMarkerTok}[1]{\textcolor[rgb]{0.00,0.23,0.31}{#1}}
\newcommand{\SpecialCharTok}[1]{\textcolor[rgb]{0.37,0.37,0.37}{#1}}
\newcommand{\SpecialStringTok}[1]{\textcolor[rgb]{0.13,0.47,0.30}{#1}}
\newcommand{\StringTok}[1]{\textcolor[rgb]{0.13,0.47,0.30}{#1}}
\newcommand{\VariableTok}[1]{\textcolor[rgb]{0.07,0.07,0.07}{#1}}
\newcommand{\VerbatimStringTok}[1]{\textcolor[rgb]{0.13,0.47,0.30}{#1}}
\newcommand{\WarningTok}[1]{\textcolor[rgb]{0.37,0.37,0.37}{\textit{#1}}}

\providecommand{\tightlist}{%
  \setlength{\itemsep}{0pt}\setlength{\parskip}{0pt}}\usepackage{longtable,booktabs,array}
\usepackage{calc} % for calculating minipage widths
% Correct order of tables after \paragraph or \subparagraph
\usepackage{etoolbox}
\makeatletter
\patchcmd\longtable{\par}{\if@noskipsec\mbox{}\fi\par}{}{}
\makeatother
% Allow footnotes in longtable head/foot
\IfFileExists{footnotehyper.sty}{\usepackage{footnotehyper}}{\usepackage{footnote}}
\makesavenoteenv{longtable}
\usepackage{graphicx}
\makeatletter
\newsavebox\pandoc@box
\newcommand*\pandocbounded[1]{% scales image to fit in text height/width
  \sbox\pandoc@box{#1}%
  \Gscale@div\@tempa{\textheight}{\dimexpr\ht\pandoc@box+\dp\pandoc@box\relax}%
  \Gscale@div\@tempb{\linewidth}{\wd\pandoc@box}%
  \ifdim\@tempb\p@<\@tempa\p@\let\@tempa\@tempb\fi% select the smaller of both
  \ifdim\@tempa\p@<\p@\scalebox{\@tempa}{\usebox\pandoc@box}%
  \else\usebox{\pandoc@box}%
  \fi%
}
% Set default figure placement to htbp
\def\fps@figure{htbp}
\makeatother

\KOMAoption{captions}{tableheading}
\makeatletter
\@ifpackageloaded{caption}{}{\usepackage{caption}}
\AtBeginDocument{%
\ifdefined\contentsname
  \renewcommand*\contentsname{Table of contents}
\else
  \newcommand\contentsname{Table of contents}
\fi
\ifdefined\listfigurename
  \renewcommand*\listfigurename{List of Figures}
\else
  \newcommand\listfigurename{List of Figures}
\fi
\ifdefined\listtablename
  \renewcommand*\listtablename{List of Tables}
\else
  \newcommand\listtablename{List of Tables}
\fi
\ifdefined\figurename
  \renewcommand*\figurename{Figure}
\else
  \newcommand\figurename{Figure}
\fi
\ifdefined\tablename
  \renewcommand*\tablename{Table}
\else
  \newcommand\tablename{Table}
\fi
}
\@ifpackageloaded{float}{}{\usepackage{float}}
\floatstyle{ruled}
\@ifundefined{c@chapter}{\newfloat{codelisting}{h}{lop}}{\newfloat{codelisting}{h}{lop}[chapter]}
\floatname{codelisting}{Listing}
\newcommand*\listoflistings{\listof{codelisting}{List of Listings}}
\makeatother
\makeatletter
\makeatother
\makeatletter
\@ifpackageloaded{caption}{}{\usepackage{caption}}
\@ifpackageloaded{subcaption}{}{\usepackage{subcaption}}
\makeatother

\usepackage{bookmark}

\IfFileExists{xurl.sty}{\usepackage{xurl}}{} % add URL line breaks if available
\urlstyle{same} % disable monospaced font for URLs
\hypersetup{
  pdftitle={programming\_in\_R},
  colorlinks=true,
  linkcolor={blue},
  filecolor={Maroon},
  citecolor={Blue},
  urlcolor={Blue},
  pdfcreator={LaTeX via pandoc}}


\title{programming\_in\_R}
\author{}
\date{}

\begin{document}
\maketitle


\subsection{Task One: Conceptual
Questions}\label{task-one-conceptual-questions}

\subsubsection{\texorpdfstring{Question 1: What is the purpose of the
\texttt{lapply()} function? What is the equivalent \texttt{purrr}
function?}{Question 1: What is the purpose of the lapply() function? What is the equivalent purrr function?}}\label{question-1-what-is-the-purpose-of-the-lapply-function-what-is-the-equivalent-purrr-function}

The \texttt{lapply()} function is used for list objects. It applies the
specified function to every element of the list and then returns a list.
The equivalent \texttt{purrr} function is \texttt{map()}.

\subsubsection{\texorpdfstring{Question 2: Suppose we have a list called
\texttt{my\_list}. Each element of the list is a numeric data frame (all
columns}{Question 2: Suppose we have a list called my\_list. Each element of the list is a numeric data frame (all columns}}\label{question-2-suppose-we-have-a-list-called-my_list.-each-element-of-the-list-is-a-numeric-data-frame-all-columns}

are numeric). We want use \texttt{lapply()} to run the code
\texttt{cor(numeric\_matrix,\ method\ =\ "kendall")} on each element of
the list. Write code to do this below! (I'm really trying to ask you how
you specify \texttt{method\ =\ "kendall"} when calling
\texttt{lapply()})

\begin{Shaded}
\begin{Highlighting}[]
\CommentTok{\# lapply(X = my\_list, }
\CommentTok{\#        FUN = cor,}
\CommentTok{\#        method = "kendall")}
\end{Highlighting}
\end{Shaded}

\subsubsection{\texorpdfstring{Question 3: What are two advantages of
using \texttt{purrr} functions instead of the \texttt{BaseR} apply
family?}{Question 3: What are two advantages of using purrr functions instead of the BaseR apply family?}}\label{question-3-what-are-two-advantages-of-using-purrr-functions-instead-of-the-baser-apply-family}

One advantages of the \texttt{purrr} functions is that they are
generally more consistent (additional arguments can be specified under
the same naming conventions, the order of arguments is consistent,
etc.). Another advantage is that the \texttt{purrr} functions contain
useful helper functions (like \texttt{partial()}) and variants (like
\texttt{map2()}).

\subsubsection{Question 4: What is a side-effect
function?}\label{question-4-what-is-a-side-effect-function}

A side-effect function uses the data to perform a task, but when the
task is complete it returns the original data unmodified.

\subsubsection{\texorpdfstring{Question 5: Why can you name a variable
\texttt{sd} in a function and not cause any issues with the \texttt{sd}
function?}{Question 5: Why can you name a variable sd in a function and not cause any issues with the sd function?}}\label{question-5-why-can-you-name-a-variable-sd-in-a-function-and-not-cause-any-issues-with-the-sd-function}

Variables that are created in functions are only stored temporarily in
the function environment. They are not kept after the function call is
finished, and therefore never stored in the global environment.

\newpage

\subsection{Task Two: Writing R
Functions}\label{task-two-writing-r-functions}

\subsubsection{Question 1: Writing RMSE
function}\label{question-1-writing-rmse-function}

\begin{Shaded}
\begin{Highlighting}[]
\CommentTok{\#Writing the RMSE function as taking in two vectors}
\NormalTok{getRMSE }\OtherTok{\textless{}{-}} \ControlFlowTok{function}\NormalTok{(resp\_vec, pred\_vec, ...) \{}
\NormalTok{  sq\_error }\OtherTok{\textless{}{-}}\NormalTok{ (resp\_vec }\SpecialCharTok{{-}}\NormalTok{ pred\_vec)}\SpecialCharTok{\^{}}\DecValTok{2} \CommentTok{\#finding square error}
\NormalTok{  mean }\OtherTok{\textless{}{-}} \FunctionTok{mean}\NormalTok{(sq\_error, ...) }\CommentTok{\#finding mean square error (MSE)}
\NormalTok{  rmse }\OtherTok{\textless{}{-}} \FunctionTok{sqrt}\NormalTok{(mean) }\CommentTok{\#finding root MSE}
  \FunctionTok{return}\NormalTok{(rmse) }\CommentTok{\#returning the calculated RMSE value}
\NormalTok{  \}}
\end{Highlighting}
\end{Shaded}

\subsubsection{Question 2: Testing RMSE
function}\label{question-2-testing-rmse-function}

\begin{Shaded}
\begin{Highlighting}[]
\CommentTok{\#Generating test data}
\FunctionTok{set.seed}\NormalTok{(}\DecValTok{10}\NormalTok{) }\CommentTok{\#setting random seed so that data can be replicated}
\NormalTok{n }\OtherTok{\textless{}{-}} \DecValTok{100} \CommentTok{\#we want 100 generated data points}
\NormalTok{x }\OtherTok{\textless{}{-}} \FunctionTok{runif}\NormalTok{(n) }\CommentTok{\#randomly generating 100 data points from the uniform distribution}
\NormalTok{resp }\OtherTok{\textless{}{-}} \DecValTok{3} \SpecialCharTok{+} \DecValTok{10}\SpecialCharTok{*}\NormalTok{x }\SpecialCharTok{+} \FunctionTok{rnorm}\NormalTok{(n) }\CommentTok{\#generating response data}
\NormalTok{pred }\OtherTok{\textless{}{-}} \FunctionTok{predict}\NormalTok{(}\FunctionTok{lm}\NormalTok{(resp }\SpecialCharTok{\textasciitilde{}}\NormalTok{ x), }\FunctionTok{data.frame}\NormalTok{(x)) }\CommentTok{\#generating prediction data}
\end{Highlighting}
\end{Shaded}

Testing my RMSE function

\begin{Shaded}
\begin{Highlighting}[]
\FunctionTok{getRMSE}\NormalTok{(resp, pred)}
\end{Highlighting}
\end{Shaded}

\begin{verbatim}
[1] 0.9581677
\end{verbatim}

Replacing several observations with missing values

\begin{Shaded}
\begin{Highlighting}[]
\NormalTok{resp[}\DecValTok{3}\NormalTok{] }\OtherTok{\textless{}{-}} \ConstantTok{NA\_real\_} \CommentTok{\#setting the third response to missing}
\NormalTok{resp[}\DecValTok{55}\NormalTok{] }\OtherTok{\textless{}{-}} \ConstantTok{NA\_real\_} \CommentTok{\#setting the 55th response to missing}
\end{Highlighting}
\end{Shaded}

Testing my RMSE function without specifying missing values behavior

\begin{Shaded}
\begin{Highlighting}[]
\FunctionTok{getRMSE}\NormalTok{(resp, pred)}
\end{Highlighting}
\end{Shaded}

\begin{verbatim}
[1] NA
\end{verbatim}

Testing my RMSE function with removing missing values

\begin{Shaded}
\begin{Highlighting}[]
\FunctionTok{getRMSE}\NormalTok{(resp, pred, }\AttributeTok{na.rm =} \ConstantTok{TRUE}\NormalTok{)}
\end{Highlighting}
\end{Shaded}

\begin{verbatim}
[1] 0.9568069
\end{verbatim}

\subsection{Question 3: Writing Mean Absolute Deviation
Function}\label{question-3-writing-mean-absolute-deviation-function}

\begin{Shaded}
\begin{Highlighting}[]
\CommentTok{\#Writing the MAE function as taking in two vectors}
\NormalTok{getMAE }\OtherTok{\textless{}{-}} \ControlFlowTok{function}\NormalTok{(resp\_vec, pred\_vec, ...)\{}
\NormalTok{  diff }\OtherTok{\textless{}{-}}\NormalTok{ resp\_vec }\SpecialCharTok{{-}}\NormalTok{ pred\_vec }\CommentTok{\#finding the difference between prediction and response}
\NormalTok{  abs\_diff }\OtherTok{\textless{}{-}} \FunctionTok{abs}\NormalTok{(diff) }\CommentTok{\#finding the absolute difference}
\NormalTok{  mae }\OtherTok{\textless{}{-}} \FunctionTok{mean}\NormalTok{(abs\_diff, ...) }\CommentTok{\#finding MAE}
  \FunctionTok{return}\NormalTok{(mae) }\CommentTok{\#returning MAE}
\NormalTok{\}}
\end{Highlighting}
\end{Shaded}

\subsection{Question 4: Testing MAE
Function}\label{question-4-testing-mae-function}

Generating Test Data

\begin{Shaded}
\begin{Highlighting}[]
\FunctionTok{set.seed}\NormalTok{(}\DecValTok{10}\NormalTok{) }\CommentTok{\#setting random seed so that data can be replicated}
\NormalTok{n }\OtherTok{\textless{}{-}} \DecValTok{100} \CommentTok{\#we want 100 generated data points}
\NormalTok{x }\OtherTok{\textless{}{-}} \FunctionTok{runif}\NormalTok{(n) }\CommentTok{\#randomly generating 100 data points from the uniform distribution}
\NormalTok{resp }\OtherTok{\textless{}{-}} \DecValTok{3} \SpecialCharTok{+} \DecValTok{10}\SpecialCharTok{*}\NormalTok{x }\SpecialCharTok{+} \FunctionTok{rnorm}\NormalTok{(n) }\CommentTok{\#generating response data}
\NormalTok{pred }\OtherTok{\textless{}{-}} \FunctionTok{predict}\NormalTok{(}\FunctionTok{lm}\NormalTok{(resp }\SpecialCharTok{\textasciitilde{}}\NormalTok{ x), }\FunctionTok{data.frame}\NormalTok{(x)) }\CommentTok{\#generating prediction data}
\end{Highlighting}
\end{Shaded}

Testing my MAE function

\begin{Shaded}
\begin{Highlighting}[]
\FunctionTok{getMAE}\NormalTok{(resp, pred)}
\end{Highlighting}
\end{Shaded}

\begin{verbatim}
[1] 0.8155776
\end{verbatim}

Replacing several observations with missing values

\begin{Shaded}
\begin{Highlighting}[]
\NormalTok{resp[}\DecValTok{3}\NormalTok{] }\OtherTok{\textless{}{-}} \ConstantTok{NA\_real\_} \CommentTok{\#setting the third response to missing}
\NormalTok{resp[}\DecValTok{55}\NormalTok{] }\OtherTok{\textless{}{-}} \ConstantTok{NA\_real\_} \CommentTok{\#setting the 55th response to missing}
\end{Highlighting}
\end{Shaded}

Testing my MAE function without specifying missing values behavior

\begin{Shaded}
\begin{Highlighting}[]
\FunctionTok{getMAE}\NormalTok{(resp, pred)}
\end{Highlighting}
\end{Shaded}

\begin{verbatim}
[1] NA
\end{verbatim}

Testing my MAE function with removing missing values

\begin{Shaded}
\begin{Highlighting}[]
\FunctionTok{getMAE}\NormalTok{(resp, pred, }\AttributeTok{na.rm =} \ConstantTok{TRUE}\NormalTok{)}
\end{Highlighting}
\end{Shaded}

\begin{verbatim}
[1] 0.812853
\end{verbatim}

\subsubsection{Question 5: Writing a Wrapper Function to Calculate RMSE
and
MAE}\label{question-5-writing-a-wrapper-function-to-calculate-rmse-and-mae}

\begin{Shaded}
\begin{Highlighting}[]
\NormalTok{wrapper }\OtherTok{\textless{}{-}} \ControlFlowTok{function}\NormalTok{(resp\_vec, pred\_vec, }\AttributeTok{metric =} \FunctionTok{c}\NormalTok{(}\StringTok{"RMSE"}\NormalTok{, }\StringTok{"MAE"}\NormalTok{), ...) \{}
  \ControlFlowTok{if}\NormalTok{ (}\SpecialCharTok{!}\FunctionTok{is.vector}\NormalTok{(resp\_vec) }\SpecialCharTok{|} \SpecialCharTok{!}\FunctionTok{is.vector}\NormalTok{(pred\_vec) }\SpecialCharTok{|} \CommentTok{\#checking vectors}
      \SpecialCharTok{!}\FunctionTok{is.numeric}\NormalTok{(resp\_vec) }\SpecialCharTok{|} \SpecialCharTok{!}\FunctionTok{is.numeric}\NormalTok{(pred\_vec) }\SpecialCharTok{|} \CommentTok{\#checking numeric}
      \SpecialCharTok{!}\FunctionTok{is.atomic}\NormalTok{(resp\_vec) }\SpecialCharTok{|} \SpecialCharTok{!}\FunctionTok{is.atomic}\NormalTok{(pred\_vec)) \{ }\CommentTok{\#checking atomic}
    \FunctionTok{print}\NormalTok{(}\StringTok{"ERROR: the responses and/or the predictions are not a numeric(atomic) vector"}\NormalTok{)}
\NormalTok{    \} }\ControlFlowTok{else} \ControlFlowTok{if}\NormalTok{ (}\FunctionTok{all}\NormalTok{(}\FunctionTok{c}\NormalTok{(}\StringTok{"RMSE"}\NormalTok{,}\StringTok{"MAE"}\NormalTok{) }\SpecialCharTok{\%in\%}\NormalTok{ metric)) \{ }\CommentTok{\#checking for both metrics}
\NormalTok{      rmse }\OtherTok{\textless{}{-}} \FunctionTok{getRMSE}\NormalTok{(resp\_vec, pred\_vec, ...) }\CommentTok{\#calculating RMSE}
\NormalTok{      mae }\OtherTok{\textless{}{-}} \FunctionTok{getMAE}\NormalTok{(resp\_vec, pred\_vec, ...) }\CommentTok{\#calculating MAE}
      \FunctionTok{return}\NormalTok{(}\FunctionTok{c}\NormalTok{(}\FunctionTok{paste}\NormalTok{(}\StringTok{"RMSE ="}\NormalTok{, rmse), }\FunctionTok{paste}\NormalTok{(}\StringTok{"MAE ="}\NormalTok{, mae)))}
\NormalTok{      \} }\ControlFlowTok{else} \ControlFlowTok{if}\NormalTok{ (metric }\SpecialCharTok{==} \StringTok{"rmse"} \SpecialCharTok{|}\NormalTok{ metric }\SpecialCharTok{==} \StringTok{"RMSE"}\NormalTok{) \{ }\CommentTok{\#checking for RMSE}
\NormalTok{        rmse }\OtherTok{\textless{}{-}} \FunctionTok{getRMSE}\NormalTok{(resp\_vec, pred\_vec, ...) }\CommentTok{\#calculating RMSE}
        \FunctionTok{return}\NormalTok{(}\FunctionTok{paste}\NormalTok{(}\StringTok{"RMSE ="}\NormalTok{, rmse))}
\NormalTok{        \} }\ControlFlowTok{else} \ControlFlowTok{if}\NormalTok{ (metric }\SpecialCharTok{==} \StringTok{"mae"} \SpecialCharTok{|}\NormalTok{ metric }\SpecialCharTok{==} \StringTok{"MAE"}\NormalTok{) \{ }\CommentTok{\#checking for MAE}
\NormalTok{          mae }\OtherTok{\textless{}{-}} \FunctionTok{getMAE}\NormalTok{(resp\_vec, pred\_vec, ...) }\CommentTok{\#calculating MAE}
          \FunctionTok{return}\NormalTok{(}\FunctionTok{paste}\NormalTok{(}\StringTok{"MAE ="}\NormalTok{, mae))}
\NormalTok{        \}}
\NormalTok{  \}}
\end{Highlighting}
\end{Shaded}

\subsubsection{Question 6: Testing the Wrapper
Function}\label{question-6-testing-the-wrapper-function}

\begin{Shaded}
\begin{Highlighting}[]
\CommentTok{\#Generating Test Data}
\FunctionTok{set.seed}\NormalTok{(}\DecValTok{10}\NormalTok{) }\CommentTok{\#setting random seed so that data can be replicated}
\NormalTok{n }\OtherTok{\textless{}{-}} \DecValTok{100} \CommentTok{\#we want 100 generated data points}
\NormalTok{x }\OtherTok{\textless{}{-}} \FunctionTok{runif}\NormalTok{(n) }\CommentTok{\#randomly generating 100 data points from the uniform distribution}
\NormalTok{resp }\OtherTok{\textless{}{-}} \DecValTok{3} \SpecialCharTok{+} \DecValTok{10}\SpecialCharTok{*}\NormalTok{x }\SpecialCharTok{+} \FunctionTok{rnorm}\NormalTok{(n) }\CommentTok{\#generating response data}
\NormalTok{pred }\OtherTok{\textless{}{-}} \FunctionTok{predict}\NormalTok{(}\FunctionTok{lm}\NormalTok{(resp }\SpecialCharTok{\textasciitilde{}}\NormalTok{ x), }\FunctionTok{data.frame}\NormalTok{(x)) }\CommentTok{\#generating prediction data}
\end{Highlighting}
\end{Shaded}

Testing the Wrapper Function Under Metric Default (Both)

\begin{Shaded}
\begin{Highlighting}[]
\FunctionTok{wrapper}\NormalTok{(resp, pred)}
\end{Highlighting}
\end{Shaded}

\begin{verbatim}
[1] "RMSE = 0.958167655151933" "MAE = 0.815577593682669" 
\end{verbatim}

Testing the Wrapper Function Specifying RMSE as the Metric

\begin{Shaded}
\begin{Highlighting}[]
\FunctionTok{wrapper}\NormalTok{(resp, pred, }\StringTok{"rmse"}\NormalTok{)}
\end{Highlighting}
\end{Shaded}

\begin{verbatim}
[1] "RMSE = 0.958167655151933"
\end{verbatim}

Testing the Wrapper Function Specifying MAE as the Metric

\begin{Shaded}
\begin{Highlighting}[]
\FunctionTok{wrapper}\NormalTok{(resp, pred, }\StringTok{"mae"}\NormalTok{)}
\end{Highlighting}
\end{Shaded}

\begin{verbatim}
[1] "MAE = 0.815577593682669"
\end{verbatim}

Replacing several observations with missing values

\begin{Shaded}
\begin{Highlighting}[]
\NormalTok{resp[}\DecValTok{3}\NormalTok{] }\OtherTok{\textless{}{-}} \ConstantTok{NA\_real\_} \CommentTok{\#setting the third response to missing}
\NormalTok{resp[}\DecValTok{55}\NormalTok{] }\OtherTok{\textless{}{-}} \ConstantTok{NA\_real\_} \CommentTok{\#setting the 55th response to missing}
\end{Highlighting}
\end{Shaded}

Testing the Wrapper Function Under Metric Default (Both) with Missing
Data

\begin{Shaded}
\begin{Highlighting}[]
\FunctionTok{wrapper}\NormalTok{(resp, pred)}
\end{Highlighting}
\end{Shaded}

\begin{verbatim}
[1] "RMSE = NA" "MAE = NA" 
\end{verbatim}

Testing the Wrapper Function Under Metric Default (Both) with Missing
Data Removed

\begin{Shaded}
\begin{Highlighting}[]
\FunctionTok{wrapper}\NormalTok{(resp, pred, }\AttributeTok{na.rm =} \ConstantTok{TRUE}\NormalTok{)}
\end{Highlighting}
\end{Shaded}

\begin{verbatim}
[1] "RMSE = 0.956806850171415" "MAE = 0.812853024472129" 
\end{verbatim}

Testing the Wrapper Function Specifying RMSE as the Metric with Missing
Data

\begin{Shaded}
\begin{Highlighting}[]
\FunctionTok{wrapper}\NormalTok{(resp, pred, }\StringTok{"rmse"}\NormalTok{)}
\end{Highlighting}
\end{Shaded}

\begin{verbatim}
[1] "RMSE = NA"
\end{verbatim}

Testing the Wrapper Function Specifying RMSE as the Metric with Missing
Data Removed

\begin{Shaded}
\begin{Highlighting}[]
\FunctionTok{wrapper}\NormalTok{(resp, pred, }\StringTok{"rmse"}\NormalTok{, }\AttributeTok{na.rm =} \ConstantTok{TRUE}\NormalTok{)}
\end{Highlighting}
\end{Shaded}

\begin{verbatim}
[1] "RMSE = 0.956806850171415"
\end{verbatim}

Testing the Wrapper Function Specifying MAE as the Metric with Missing
Data

\begin{Shaded}
\begin{Highlighting}[]
\FunctionTok{wrapper}\NormalTok{(resp, pred, }\StringTok{"mae"}\NormalTok{)}
\end{Highlighting}
\end{Shaded}

\begin{verbatim}
[1] "MAE = NA"
\end{verbatim}

Testing the Wrapper Function Specifying MAE as the Metric with Missing
Data Removed

\begin{Shaded}
\begin{Highlighting}[]
\FunctionTok{wrapper}\NormalTok{(resp, pred, }\StringTok{"mae"}\NormalTok{, }\AttributeTok{na.rm =} \ConstantTok{TRUE}\NormalTok{)}
\end{Highlighting}
\end{Shaded}

\begin{verbatim}
[1] "MAE = 0.812853024472129"
\end{verbatim}

Testing the Wrapper Function When Both Arguments are not Vectors

\begin{Shaded}
\begin{Highlighting}[]
\NormalTok{resp\_df }\OtherTok{\textless{}{-}} \FunctionTok{as.data.frame}\NormalTok{(resp) }\CommentTok{\#converting responses to a data frame}
\NormalTok{pred\_df }\OtherTok{\textless{}{-}} \FunctionTok{as.data.frame}\NormalTok{(pred) }\CommentTok{\#converting predictions to a data frame}

\FunctionTok{wrapper}\NormalTok{(resp\_df, pred\_df)}
\end{Highlighting}
\end{Shaded}

\begin{verbatim}
[1] "ERROR: the responses and/or the predictions are not a numeric(atomic) vector"
\end{verbatim}

Testing the Wrapper Function When Response Argument is not a Vector

\begin{Shaded}
\begin{Highlighting}[]
\FunctionTok{wrapper}\NormalTok{(resp, pred\_df)}
\end{Highlighting}
\end{Shaded}

\begin{verbatim}
[1] "ERROR: the responses and/or the predictions are not a numeric(atomic) vector"
\end{verbatim}

Testing the Wrapper Function When Prediction Argument is not a Vector

\begin{Shaded}
\begin{Highlighting}[]
\FunctionTok{wrapper}\NormalTok{(resp\_df, pred)}
\end{Highlighting}
\end{Shaded}

\begin{verbatim}
[1] "ERROR: the responses and/or the predictions are not a numeric(atomic) vector"
\end{verbatim}

\newpage

\subsection{Task Three: Querying an API and a Tidy-Style
Function}\label{task-three-querying-an-api-and-a-tidy-style-function}

This information sometimes get cut off, but my API key is
0617778f612b4946859a211593d3efda.

\subsubsection{\texorpdfstring{Question 1: Using \texttt{GET} from the
\texttt{httr} Package to Pull News
Info}{Question 1: Using GET from the httr Package to Pull News Info}}\label{question-1-using-get-from-the-httr-package-to-pull-news-info}

\begin{Shaded}
\begin{Highlighting}[]
\CommentTok{\#Loading required packages}
\FunctionTok{library}\NormalTok{(}\StringTok{"httr"}\NormalTok{)}

\CommentTok{\#Using httr::GET to collect information about the Pacers in the news}
\NormalTok{pacers\_info }\OtherTok{\textless{}{-}}\NormalTok{ httr}\SpecialCharTok{::}\FunctionTok{GET}\NormalTok{(}\StringTok{"https://newsapi.org/v2/everything?q=pacers\&from=2025{-}06{-}22\&apiKey=0617778f612b4946859a211593d3efda"}\NormalTok{)}
\NormalTok{pacers\_info}
\end{Highlighting}
\end{Shaded}

\begin{verbatim}
Response [https://newsapi.org/v2/everything?q=pacers&from=2025-06-22&apiKey=0617778f612b4946859a211593d3efda]
  Date: 2025-06-24 21:17
  Status: 200
  Content-Type: application/json; charset=utf-8
  Size: 83.1 kB
\end{verbatim}

\subsubsection{Question 2: Parsing the News Info to Pull
Articles}\label{question-2-parsing-the-news-info-to-pull-articles}

\begin{Shaded}
\begin{Highlighting}[]
\CommentTok{\#Loading required packages}
\FunctionTok{library}\NormalTok{(}\StringTok{"tidyverse"}\NormalTok{)}
\FunctionTok{library}\NormalTok{(}\StringTok{"jsonlite"}\NormalTok{)}

\CommentTok{\#Parsing the news information about the Pacers}
\NormalTok{parsed\_pacers\_info }\OtherTok{\textless{}{-}} \FunctionTok{fromJSON}\NormalTok{(}\FunctionTok{rawToChar}\NormalTok{(pacers\_info}\SpecialCharTok{$}\NormalTok{content))}
\NormalTok{pacers\_data }\OtherTok{\textless{}{-}} \FunctionTok{as\_tibble}\NormalTok{(parsed\_pacers\_info}\SpecialCharTok{$}\NormalTok{articles) }\CommentTok{\#pulling articles}
\NormalTok{pacers\_data }\CommentTok{\#displaying the article information}
\end{Highlighting}
\end{Shaded}

\begin{verbatim}
# A tibble: 96 x 8
   source$id $name author title description url   urlToImage publishedAt content
   <chr>     <chr> <chr>  <chr> <chr>       <chr> <chr>      <chr>       <chr>  
 1 <NA>      NPR   Becky~ Afte~ Led by poi~ http~ https://n~ 2025-06-23~ "The O~
 2 espn      ESPN  ESPN ~ Jale~ The respec~ http~ https://a~ 2025-06-22~ "It's ~
 3 espn      ESPN  NBA i~ Game~ The Pacers~ http~ https://a~ 2025-06-22~ "Jun 2~
 4 espn      ESPN  David~ Game~ Multiple s~ http~ https://a~ 2025-06-22~ "The I~
 5 abc-news  ABC ~ TIM R~ Thun~ Shai Gilge~ http~ https://i~ 2025-06-23~ "OKLAH~
 6 abc-news  ABC ~ ABC N~ WATC~ The Oklaho~ http~ https://i~ 2025-06-23~ "<ul><~
 7 le-monde  Le M~ Valen~ NBA ~ Portée par~ http~ https://i~ 2025-06-23~ "Alex ~
 8 <NA>      HYPE~ info@~ The ~ SummaryThe~ http~ https://i~ 2025-06-23~ "Summa~
 9 <NA>      CNET  Matt ~ NBA ~ Discover t~ http~ https://w~ 2025-06-22~ "The N~
10 die-zeit  Die ~ ZEIT ~ Bask~ Hier finde~ http~ https://i~ 2025-06-23~ "Natio~
# i 86 more rows
\end{verbatim}

\subsubsection{Question 3: Writing a Function to Query the
API}\label{question-3-writing-a-function-to-query-the-api}

\begin{Shaded}
\begin{Highlighting}[]
\NormalTok{api\_query }\OtherTok{\textless{}{-}} \ControlFlowTok{function}\NormalTok{(subject, time, key)\{}
\NormalTok{  url }\OtherTok{=} \FunctionTok{paste0}\NormalTok{(}\StringTok{"https://newsapi.org/v2/everything?q="}\NormalTok{, }\CommentTok{\#base of URL}
\NormalTok{               subject, }\CommentTok{\#adding subject to URL}
               \StringTok{"\&from="}\NormalTok{, time, }\CommentTok{\#adding time to URL}
               \StringTok{"\&apiKey="}\NormalTok{, key) }\CommentTok{\#adding API key to URL}
\NormalTok{  subject\_info }\OtherTok{\textless{}{-}}\NormalTok{ httr}\SpecialCharTok{::}\FunctionTok{GET}\NormalTok{(url) }\CommentTok{\#extracting news info}
\NormalTok{  parsed\_subject\_info }\OtherTok{\textless{}{-}} \FunctionTok{fromJSON}\NormalTok{(}\FunctionTok{rawToChar}\NormalTok{(subject\_info}\SpecialCharTok{$}\NormalTok{content)) }\CommentTok{\#parsing content of news}
\NormalTok{  subject\_data }\OtherTok{\textless{}{-}} \FunctionTok{as\_tibble}\NormalTok{(parsed\_subject\_info}\SpecialCharTok{$}\NormalTok{articles) }\CommentTok{\#pulling articles}
  \FunctionTok{return}\NormalTok{(subject\_data)}
\NormalTok{\}}
\end{Highlighting}
\end{Shaded}

Testing my Short Function to Query the API

\begin{Shaded}
\begin{Highlighting}[]
\FunctionTok{api\_query}\NormalTok{(}\StringTok{"gamestop"}\NormalTok{, }\StringTok{"2025{-}05{-}24"}\NormalTok{, }\StringTok{"0617778f612b4946859a211593d3efda"}\NormalTok{)}
\end{Highlighting}
\end{Shaded}

\begin{verbatim}
# A tibble: 99 x 8
   source$id $name author title description url   urlToImage publishedAt content
   <chr>     <chr> <chr>  <chr> <chr>       <chr> <chr>      <chr>       <chr>  
 1 the-verge The ~ David~ A ni~ I'm standi~ http~ https://p~ 2025-06-05~ "Body ~
 2 the-verge The ~ Brand~ The ~ Amazon’s m~ http~ https://p~ 2025-06-20~ "Amazo~
 3 business~ Busi~ fdemo~ Game~ GameStop a~ http~ https://i~ 2025-05-28~ "GameS~
 4 <NA>      Gizm~ Kyle ~ Targ~ Check to m~ http~ https://g~ 2025-06-03~ "The S~
 5 <NA>      Slas~ msmash Game~ GameStop i~ http~ https://a~ 2025-06-13~ "Cohen~
 6 <NA>      Yaho~ Brad ~ Bitc~ The 2025 a~ http~ https://s~ 2025-05-28~ "Bitco~
 7 <NA>      Gizm~ James~ Did ~ Maybe orde~ http~ https://g~ 2025-06-05~ "When ~
 8 <NA>      Hipe~ Gabri~ Desa~ El gran dí~ http~ https://i~ 2025-06-05~ "El gr~
 9 <NA>      Kota~ Ethan~ Stat~ Imagine yo~ http~ https://i~ 2025-06-05~ "Imagi~
10 <NA>      Gizm~ James~ Some~ There's on~ http~ https://g~ 2025-06-18~ "The S~
# i 89 more rows
\end{verbatim}




\end{document}
